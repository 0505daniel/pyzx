\documentclass[a4paper,11pt]{article}

\linespread{1.1}
\usepackage[top=3cm, bottom=2.54cm, left=2.54cm, right=2.54cm]{geometry}
\usepackage[utf8]{inputenc}
\usepackage[english]{babel}
\usepackage{amsmath}
\usepackage{amsfonts}
\usepackage{amssymb}
\usepackage{graphicx}
\usepackage{hyperref}
\usepackage{xcolor}

\begin{document}
\pagenumbering{gobble}

\Large
 \begin{center}
T-count optimization in quantum circuits using graph-theoretical rewriting of ZX-diagrams\\ 

\hspace{10pt}

\normalsize
% Author names and affiliations
Aleks Kissinger $^1$, John van de Wetering $^1$ \\

\hspace{10pt}

\small  
$^1$ Radboud University Nijmegen, The Netherlands


\end{center}

\hspace{10pt}

\normalsize

\noindent

Most fault-tolerant architectures for quantum computers allow fast execution of Clifford gates, so that the majority of their resources is spend on implementing non-Clifford gates, specifically \emph{T-gates}. It is therefore desirable to find optimizations of quantum circuits where the \emph{T-count}, the amount of T-gates, is as low as possible. The field of T-count optimization has been quite active in the recent years. Some approaches synthesize provably optimal circuits, but these run in exponential time and therefore only work for small circuits \cite{di2016parallelizing}. Besides these algorithms there are a variety of efficient heuristic algorithms. There are for instance methods that optimize CNOT+T subcircuits using the \emph{phase-polynomial} formalism like \emph{T-par} \cite{amy2014polynomial}. This method can be made more efficient by preprocessing the hadamard gates by optimizing them \cite{abdessaied2014quantum} or `gadgetizing' them into ancillae qubits \cite{heyfron2018efficient}. A recent paper used phase polynomial techniques in combination with a variety of peephole optimizations to achieve state-of-the-art results \cite{nam2018automated}.

For T-count optimisation, we take a wildly different approach. We translate quantum circuits into \emph{ZX-diagrams}. These diagrams are a type of tensor network consisting of non-unitary generators, called Z- and X-\emph{spiders}, which come with a set of rewrite rules known as the \emph{ZX-calculus}. It has been shown that the ZX-calculus is complete for Clifford circuits, meaning that two Clifford circuits are equal if and only if one can be rewritten into the other by the rules of the ZX-calculus \cite{backens2016simplified}. There are also various known additional rewrite rules that make the calculus complete for Clifford+T \cite{jeandel2018complete} and all \cite{HarnyCompleteness} circuits. We use a specific set of rewrite rules based on the graph-theoretic notions of \emph{local complementation} and \emph{pivoting} that always terminate in finite time, in addition to a procedure called \emph{gadgetization} that rewrites T-gates into a form more amenable to simplification. The result is a ZX-diagram with a T-count that in most cases matches, and in some cases surpasses the best known state-of-the-art T-count (e.g.~in one particular case we achieve a T-count equal to 50\% of the previous best-known). The diagram produced by our rewrites is not circuit-like. We use techniques based on the notion of \emph{cut-rank} to cut our diagram into pieces that do resemble a circuit, and in this way we can extract a circuit from the ZX-diagram. The resulting circuit is not optimized for general gate-count, but post-processing with some trivial circuit identities followed by phase-polynomial optimization gives competitive values for CNOT-count and Hadamard-count.

We will present the theory behind our simplification strategy and give a demonstration of the open source Python library in which we have implemented it: \emph{PyZX}~\cite{PyZX}. The reader is also invited to watch our short demonstration video about PyZX included with the abstract.

\newpage

\bibliographystyle{plain}
\bibliography{bibliography}
\end{document}

